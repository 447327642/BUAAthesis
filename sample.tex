\documentclass[bachelor]{buaathesis}

%-------------------------------------------
%------        用户公共信息            -------
\thesisauthor{作者}{engname}
\thesistitle{论文标题}{engtitle}
\school{学院}{xueyuan}
\major{专业}{zhuanye}%专业
\teacher{指导老师}{laoshi}
\category{TP312}%中图分类号
\mibao{密保等级}%本科生此项必须写“不涉密”或为空
\thesisbegin{2011}{1}{1}%本科生为毕设开始时间;研究生为学习开始时间
\thesisend{2012}{2}{2}%本科生为毕设结束时间;研究生为学习结束时间
\defense{2012}{07}{05}%毕设答辩时间
\ckeyword{北航开源俱乐部,\LaTeX{},论文}
\ekeyword{BHOSC,\LaTeX{},Thesis}

%--------------本科生的信息--------------------
\class{3702}%班级,在任务书里需要用到
\studentID{233}%学号
\unicode{8606}
\thesisdate{2012}{6}%论文时间,用于首页

%--------------研究生的信息--------------------
\direction{研究方向}%研究方向
\teacherdegree{老师的等级}{engdegree}%需要英文?
\thesisID{0000}%论文编号
\commit{2012}{3}{3}%论文提交时间
\award{2012}{4}{4}%学位授予日期,好奇葩...

%--------------------------------------------


\begin{document}
\maketitle
\tableofcontents
\begin{cabstract}
这里是中文摘要部分。
\end{cabstract}
\begin{eabstract}
Here is the Abstract in English.
\end{eabstract}
\mainmatter
\chapter{简介}
	\section{版权声明}
	这里是奇数页\par
	这里是段首\hfill 这里是行尾\\
	这里是行首\hfill 这里是行尾\\
	\section{免责声明}
	\newpage
	\section{版本历史}
	这里是偶数页\par
	这里是段首\hfill 这里是行尾\\
	这里是行首\hfill 这里是行尾\\
	
\chapter{下载和使用}
	\section{发行版本}
	\section{开发版本}
	\section{如何使用}
	
\chapter{软件和环境的配置}
	\section{Windows用户}
	\section{Linux用户}
	\section{Mac用户}
	
\chapter{使用说明}
	\section{基本范例}
	\texttt{documentclass[master,oneside]\{buaathesis\}}
	\newpage
	\section{选项}
参数为\texttt{bachelor/master/engineer/doctor;openright/openany; oneside/twoside}。
默认参数为\texttt{master,openright,twoside}。\par
前面四个参数就不说了,\texttt{openright}会使新的章在奇数页开始,\texttt{openany}则只会新起一页,
并不判断是奇数或偶数页;\texttt{oneside/twoside}则为单面/双面打印,“双面打印”时,左右边距会根据奇偶页变化。
\chapter{LaTeX{}基础语法}
	\section{字体字号设置}
	\section{章节标题设置}
	
\chapter{模板代码实现}
%此章主要为对本模板进行代码讲解,以供有兴趣的同学进行研究
	\section{选项和编译}
	\section{页面设置}
	\newpage
	\section{正文前的部分}
	\section{正文}
		\subsection{图形和多个图形}
		\subsection{表格和长表格}
		\subsection{定理和公式}
	\newpage
	\section{参考文献}
	\section{正文之后的部分}
	
\chapter{致谢}
\cleardoublepage
%\renewcommand\refname{参考文献}
%\bibliography{bibs}
%\addcontentsline{toc}{chapter}{参考文献}
%\cleardoublepage

\appendix

%\backmatter
%backmatter会使后面的附录没有“附录A”之类的字样
%这三个命令的具体定义可见http://m.lilacbbs.com/bbstcon.php?board=TeX&gid=2269

\chapter{常见问题}

\chapter{联系我们}
\end{document}