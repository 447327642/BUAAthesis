\chapter{软件和环境的配置}
	\section{Windows用户}
		\subsection{使用\TeX{}live 2011+\TeX{}maker}
		\TeX{}是一个自由的软件,所以它有很多种发行版本。每个发行版本都是一整套工具的集合。其中一般包括了plainTeX,LaTeX,pdfTeX,dvips等。尽管有许多的发生版本,但比较主流的还属\TeX{}live,您可以在\url{http://www.tug.org/texlive/}找到最新的版本的相应的下载链接,当然也可以从北航FGBT上搜索下载。\par
		下载得到相应的iso文件,使用虚拟光驱软件加载后,直接双击运行,或者进行光盘里双击\textsl{install-tl.bat}运行。按照向导一步一步的安装即可。整个安装过程大概10$\sim$20分钟。\par
		\TeX{}maker则是一款windows下比较方便和友好的\TeX{}编辑器,在\url{http://www.xm1math.net/texmaker/}可以找到相应下载。\par
		按照向导安装完\TeX{}maker后,选择菜单栏“选项”->"配置Texmaker",在“LaTeX”一栏中填写\texttt{xelatex -interaction=nonstopmode \%.tex}~即可。其余的配置请根据自己的需要进行配置。\par
		\subsection{使用C\TeX{}套装}
		C\TeX{}中文套装是基于windows下的MiKTeX系统,集成了编辑器WinEdt和PostScript处理软件Ghostscript和GSview等主要工具。C\TeX{}中文套装在MiKTeX的基础上增加了对中文的完整支持,并且其自带的ctex宏包重构了符合中文习惯的模板样式。\par
		C\TeX{}套装可以在\url{http://www.ctex.org/CTeXDownload}找到相应的下载。
		\subsection{使用其他的编译软件及编辑器}
		其他的\LaTeX{}编译的,有MikTeX,而编辑器则可以是WinEdt,TeXnicCenter,甚至是windows自带的记事本都可以。
	\section{Linux用户}
	\section{Mac用户}